
%%%
% Any line that begins with a percent symbol is a comment. To compile
% this document and view the output:
%
% Run Latex
% Run Bibtex
% Then run Latex twice.
%
% This should produce the output PDF file named main.pdf
%%%

% This defines the style to use for this document.
% Do not modify.
\documentclass[letterpaper]{article}

% The following are akin to "import" statements in Python or Java -
% these import useful commands into the document for you to use.  You
% don't have to modify any of these lines. The AAAI package formats
% this document in the style of submissions to the American
% Association for Artificial Intelligence conference, one of the top
% AI conferences in the world. You will find that many academic
% publications in AI use this format.
\usepackage{aaai} 
\usepackage{times} 
\usepackage{helvet} 
\usepackage{courier} 
\setlength{\pdfpagewidth}{8.5in} 
\setlength{\pdfpageheight}{11in} 
\usepackage{amsmath}
\usepackage{amsthm}
\usepackage{graphicx}
\usepackage{graphics}
\usepackage{moreverb}
\usepackage{subfigure}
\usepackage{epsfig}
\usepackage{txfonts}
\usepackage{algpseudocode}
\usepackage{multirow, multicol}
\usepackage{url}
\usepackage{tablefootnote}
\usepackage{color}

\setcounter{secnumdepth}{1}
\nocopyright

% Fill in your paper title, names and emails below
% The "\\" is used to break lines. The \url command
% is useful for typesetting URLs and email addresses (it uses the
% Courier font).
\title{Solving the 8-Puzzle}
 \author{Shahin Ahmadi \and DuBose Tuller\\
 \url{{shahmadi, dutuller}@davidson.edu}\\
 Davidson College\\
 Davidson, NC 28035\\
 U.S.A.}

% This is the "true" start of the document. All the text in your
% write-up should be placed within the \begin{document} and
% \end{document} decorators.
\begin{document}

\maketitle % formats the title nicely, do not modify

% While at this point you could just begin your write-up, often, it's
% useful to write each section of your write-up in a separate tex
% file (not unlike the modular decomposition you do for code you
% write). These \input commands insert the contents of the
% specified tex files in the order specified. Every write-up you
% submit must contain the following sections, in the shown order. Open
% each of the indicated tex files to understand what goes in each
% section, as well as for more TeX tips.

% Place the contents of your abstract between the
% \begin{abstract} and \end{abstract} decorators.

\begin{abstract}


The 8-puzzle is a sliding game consisting of numbered square tiles in a 3-by-3 grid with one tile absent. We implemented the A* search algorithm to solve the 8-puzzle and assessed the impact of three different heuristic functions on search performance. The Manhattan Distance heuristic resulted in the most efficient searching. We concluded heuristics designed based on less relaxed versions of the 8-puzzle lead to more efficient A* search performance.

\end{abstract}



% The \section{} command formats and sets the title of this
% section. We'll deal with labels later.
\section{Introduction}
\label{sec:intro}
%Say you're replicating a previous experimenting to assess the impact of heuristic function on A* runtime.

A* is an informed search algorithm that uses a heuristic function to find the optimal path to a goal state in a graph. The choice of heuristic function can significantly impact the efficiency of A*. A well-designed heuristic function can greatly reduce the number of nodes that need to be expanded in the search process, leading to faster runtime. Previous research by Russell and Norvig (2003) has shown that solving the 8-puzzle with a heuristic based on the Manhattan distance between each tile and its goal position is more efficient than one based on the count of misplaced tiles. Here, we assess the impact of heuristic quality on A* performance by replicating Russell and Norvig’s experiment and comparing the outcomes. We also examine and compare the effect of the checkerboard misplaced heuristic on the A* performance in solving the 8-puzzle.


\section{Background}
\label{sec:background}
The 8-puzzle is a simple sliding game consisting of 8 numbered square tiles and an empty tile placed randomly on a 3-by-3 grid. 
The game involves rearranging the numbered tiles within the frame by sliding them one at a time into the empty tile, with the goal of arranging the numbered tiles in ascending numerical order with the empty tile in the upper-left corner of the frame. \\

\noindent To solve the 8-puzzle, we used A* algorithm with three different heuristic functions:
\begin{itemize}
  \item Manhattan Distance ($h_{MD}$): calculates the sum of the Manhattan distance between each tile and its correct position in the goal state.
  \item Misplaced Tiles ($h_{MT}$): finds the number of tiles that are not in the correct position.
  \item Checkerboard Misplaced ($h_{CM}$): divides the frame into two sub-graphs composed of tiles connected by diagonals and determines the minimum number of steps between a tile's current and target position, where the tile can move to any location in the \emph{other} sub-graph.
\end{itemize}

\noindent All three heuristics are admissible and derived by creating a relaxed version of the 8-puzzle’s rules that a computer can solve without searching. Heuristics designed based on "less" relaxed problems produce estimates that are closer to the actual search cost \cite{heuristics}.



\section{Experiments}
\label{sec:expts}

% In this section, you should describe your experimental setup. What
% were the questions you were trying to answer? What was the
% experimental setup (number of trials, parameter settings, etc.)? What
% were you measuring? You should justify these choices when
% necessary. The accepted wisdom is that there should be enough detail
% in this section that I could reproduce your work \emph{exactly} if I
% were so motivated.


%DISCUSS ORGANIZATION PARAGRAPHS
We generated random 8-puzzle board states by starting from the goal state and making a specified number of random moves. Another way to achieve this would be to generate random permutations of tiles. However, we reasoned the second approach would also generate unsolvable board states and thus can be computationally more expensive. We performed A* search on randomized board states with each of the three heuristics. We evaluated each heuristic by measuring the search cost ($N$) and effective branching factor ($b^{*}$). $N$ is measured by the total number of board states generated instead of the number of explored board states, as the latter approach underestimates the computational work. $b^{*}$ estimates the average branching factor of a tree containing only the searched board states and is calculated with the following formula \cite{aima}:
$$N + 1 = 1 + b^{*} + (b^{*})^{2} + \dots + (b^{*})^{d}$$

\noindent where $d$ is the depth of the goal state (length of the optimal path). Note that $b^{*}$ cannot be calculated analytically and is instead estimated to a desired level of precision using techniques such as Newton's Method.\\

\noindent We excluded the empty tile in our implementation of $h_{MD}$ and $h_{MT}$ to maintain admissibility. We tested each heuristic against 10,000 board states generated by 100 random moves. We noticed that board states with an optimal path length of 2-10 were not generated nearly as often, so we supplemented our data by running an additional 1000 board states using 20 random moves. When searching, we did not allow for direct backtracking to the first previous board state to avoid redundant checking. Our implementation of A* entered an infinite loop when given an unsolvable board state. Finally, we grouped the search costs by their optimal distance and averaged them within each group. Though the groups did not have the same number of trials, they were all sufficiently large to be representative.

\section{Results}
\label{sec:results}

% Present the results of your experiments. Simply presenting the data is
% insufficient! You need to analyze your results. What did you discover?
% What is interesting about your results? Were the results what you
% expected? Use appropriate visualizations. Prefer graphs and charts to
% tables as they are easier to read (though tables are often more
% compact, and can be a better choice if you're squeezed for space).
The table below summarizes the average search cost of each heuristic in conjunction with A* at a given depth. Each heuristic's relative performance can be measured by their $b^{*}$ values, with lower values corresponding to lower search costs \cite{aima}. \\

\begin{figure}[!ht]
    \centering
    
    \begin{tabular}{|l||l|l|l||l|l|l|}
    \hline
        & \multicolumn{3}{|c|}{Average Search Cost} &
           \multicolumn{3}{|c|}{Branching Factor} \\ \hline
        d & $h_{MD}$ & $h_{CM}$ & $h_{MT}$ & $h_{MD}$ & $h_{CM}$ & $h_{MT}$ \\ \hline
        2 & 5 & 5 & 6 & 1.79 & 1.79 & 2.00 \\ \hline
        4 & 8 & 8 & 9 & 1.30 & 1.30 & 1.35 \\ \hline
        6 & 14 & 15 & 15 & 1.25 & 1.27 & 1.27 \\ \hline
        8 & 21 & 23 & 34 & 1.21 & 1.23 & 1.32 \\ \hline
        10 & 35 & 46 & 73 & 1.22 & 1.27 & 1.35 \\ \hline
        12 & 63 & 97 & 179 & 1.24 & 1.30 & 1.39 \\ \hline
        14 & 124 & 236 & 486 & 1.27 & 1.34 & 1.43 \\ \hline
        16 & 275 & 600 & 1,302 & 1.29 & 1.38 & 1.46 \\ \hline
        18 & 558 & 1,463 & 3,596 & 1.31 & 1.40 & 1.48 \\ \hline
        20 & 1,127 & 3,742 & 9,975 & 1.32 & 1.42 & 1.50 \\ \hline
        22 & 2,310 & 10,062 & 28,358 & 1.34 & 1.44 & 1.52 \\ \hline
        24 & 5,024 & 27,265 & 80,559 & 1.35 & 1.46 & 1.53 \\ \hline
        26 & 11,409 & 76,691 & 234,271 & 1.36 & 1.48 & 1.55 \\ \hline
        28 & 27,664 & 215,331 & 666,423 & 1.38 & 1.49 & 1.56 \\ \hline
        30 & 73,215 & 644,093 & 2,075,996 & 1.39 & 1.50 & 1.57 \\ \hline
    \end{tabular}


    \caption{Average Search Costs and Effective Branching Factor by Heuristic Method}
\end{figure}

\noindent $h_{MD}$ outperforms the other two heuristics in all search depths and is followed by $h_{CM}$, with its $b^{*}$ values falling roughly halfway between those of $h_{MD}$ and $h_{MT}$. As shown in Figure 2, at larger $d$, small perturbations in $b^{*}$ have a large effect on search cost and can make a dramatic difference in real-world performance. For example, at $d=30$, the $h_{MT}$ heuristic took on average almost 30 times longer to run than $h_{MD}$.\\

\begin{figure}
    \centering
    \includegraphics[scale=0.6]{HW1 Paper/figs/EBFs Plot.png}
    \caption{Effective Branching Factors by Heuristic}
    \label{fig:my_label}
\end{figure}

\noindent After the problems become decently complex ($d \geq 6$), the algorithm must, on average, search a higher number of neighboring board states. This increasing $b^{*}$ effect was also observed in Russell and Norvig's experiment but happened for a slightly higher $d \geq 8$ \cite{aima}. Our implementations of $h_{MD}$ and $h_{MT}$ outperform Russell and Norvig's at lower depths ($h_{MD}$: $d < 10$; $h_{MT}$: $d \leq 14$) but do worse at higher depths (See Figure 3 and 4). Though Russell and Norvig's results do not include $d=26,28,30$, we expect the trend to be the same.\\

\begin{figure}
    \centering
    \begin{tabular}{|l||l|l||l|l|}
    \hline
        & \multicolumn{2}{|c|}{Average Search Cost} &
           \multicolumn{2}{|c|}{Branching Factor} \\ \hline
        d & $h_{MD}$  & $h_{MT}$ & $h_{MD}$ &  $h_{MT}$ \\ \hline
        2 & 6 & 6 & 1.79 & 1.79 \\ \hline
        4 & 12 & 13 & 1.45 & 1.48 \\ \hline
        6 & 18 & 20 & 1.30 & 1.34 \\ \hline
        8 & 25 & 39 & 1.24 & 1.33 \\ \hline
        10 & 39 & 93 & 1.22 & 1.38 \\ \hline
        12 & 73 & 227 & 1.24 & 1.42 \\ \hline
        14 & 113 & 539 & 1.23 & 1.44 \\ \hline
        16 & 211 & 1301 & 1.25 & 1.45 \\ \hline
        18 & 363 & 3056 & 1.26 & 1.46 \\ \hline
        20 & 676 & 7276 & 1.27 & 1.47 \\ \hline
        22 & 1219 & 18094 & 1.28 & 1.48 \\ \hline
        24 & 1641 & 39135 & 1.26 & 1.48 \\ \hline
    \end{tabular}


    \caption{Average Search Costs and Effective Branching Factor by Heuristic Method from Original Study \cite{aima}}
\end{figure}

\begin{figure}
    \centering
    \includegraphics[scale=0.55]{HW1 Paper/figs/Diffs Plot.png}
    \caption{Comparing our results to the original study. Positive values reflect higher performance by our algorithm.
}
    \label{fig:my_label}
\end{figure}

\section{Conclusions}
\label{sec:concl}

In this paper, we assessed the impact of heuristic function on the performance of A* by solving the 8-puzzle utilizing Manhattan Distance, Misplaced Tiles, and Checkerboard Misplaced heuristics. Our study confirms Russell and Norwig's results, as patterns of search cost with respect to depth are consistent. Moreover, the performance of the additional Checkerboard method falls in between the two from their study. We conclude that consistent heuristics that more accurately estimate the true number of steps to the goal state result in better A* search performance. Future research can examine other heuristics and apply $h_{MD}$, $h_{MT}$, and $h_{CM}$ to boards of larger sizes. Additionally, analyzing versions of the game where multiple goal states exist (e.g., considering all boards with ascending tiles valid regardless of the empty tile's position on the board) can provide a deeper understanding of the impact of heuristic quality on A* performance. 

% We suspect that this might have to do with the way that we generated our random states, since Russell and Norvig likely generated their board states more procedurally.
 
% In this section, briefly summarize your paper --- what problem did you
% start out to study, and what did you find? What is the key result /
% take-away message? It's also traditional to suggest one or two avenues
% for further work, but this is optional.

% 
\section{Contributions}
\label{sec:contrib}

Briefly summarize the contributions of you and your partner in this
section. For example: ``A.T. wrote the gradient descent code and ran
experiments; J.vN. ran the experiments using scikit-learn's built-in
regression solver. A.T. wrote the introduction and background sections
and prepared figure \ref{fig:tex} and table
\ref{tab:example}. J.vN. wrote the experiments and results
sections. Both authors proof-read the entire document.'' I will be
looking for roughly equal contributions from both partners in both
aspects of the assignment (i.e., the programming/data
preprocessing/experimentation and writing).


% 
\section{Acknowledgements} 
\label{sec:ack} 

This section is optional. But if there are people you'd like to thank for their help with the project --- a person who contributed some insight, friends who volunteered to help out with data collection, etc. --- then this is the place to thank them. Keep it short!


% This creates the references section. Open the project1.bib file to
% see how to organize your references.
\bibliography{project1}
\bibliographystyle{aaai} % sets citation and bib style, do not modify

\end{document}
