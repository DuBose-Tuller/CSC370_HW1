
\section{Experiments}
\label{sec:expts}

% In this section, you should describe your experimental setup. What
% were the questions you were trying to answer? What was the
% experimental setup (number of trials, parameter settings, etc.)? What
% were you measuring? You should justify these choices when
% necessary. The accepted wisdom is that there should be enough detail
% in this section that I could reproduce your work \emph{exactly} if I
% were so motivated.


%DISCUSS ORGANIZATION PARAGRAPHS
We generated random 8-puzzle board states by starting from the goal state and making a specified number of random moves. Another way to achieve this would be to generate random permutations of tiles. However, we reasoned the second approach would also generate unsolvable board states and thus can be computationally more expensive. We performed A* search on randomized board states with each of the three heuristics. We evaluated each heuristic by measuring the search cost ($N$) and effective branching factor ($b^{*}$). $N$ is measured by the total number of board states generated instead of the number of explored board states, as the latter approach underestimates the computational work. $b^{*}$ estimates the average branching factor of a tree containing only the searched board states and is calculated with the following formula \cite{aima}:
$$N + 1 = 1 + b^{*} + (b^{*})^{2} + \dots + (b^{*})^{d}$$

\noindent where $d$ is the depth of the goal state (length of the optimal path). Note that $b^{*}$ cannot be calculated analytically and is instead estimated to a desired level of precision using techniques such as Newton's Method.\\

\noindent We excluded the empty tile in our implementation of $h_{MD}$ and $h_{MT}$ to maintain admissibility. We tested each heuristic against 10,000 board states generated by 100 random moves. We noticed that board states with an optimal path length of 2-10 were not generated nearly as often, so we supplemented our data by running an additional 1000 board states using 20 random moves. When searching, we did not allow for direct backtracking to the first previous board state to avoid redundant checking. Our implementation of A* entered an infinite loop when given an unsolvable board state. Finally, we grouped the search costs by their optimal distance and averaged them within each group. Though the groups did not have the same number of trials, they were all sufficiently large to be representative.